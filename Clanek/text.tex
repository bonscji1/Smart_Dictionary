\documentclass[journal]{IEEEtran}


\usepackage[czech]{babel}


% DOPLNIJICI BALICKY
\usepackage[utf8]		%	Kódování zdrojových souborů je Windows-1250
	{inputenc}					% Balíček pro nastavení kódování zdrojových souborů
\usepackage{cmap} 		% Balíček cmap zajišťuje, že PDF vytvořené `pdflatexem' je
											% plně "prohledávatelné" a "kopírovatelné"

\usepackage{amsmath}
\usepackage{caption}
\usepackage{dsfont}
\usepackage{layouts}
\usepackage{graphicx}
\usepackage{hyperref}
\usepackage{subcaption}
\usepackage{acro}

% zkratky

\DeclareAcronym{OCR}{
  short=OCR,
  long=Optical Character Recognition
}

\DeclareAcronym{IDE}{
  short=IDE,
  long=Integrated Development Environment
}

\DeclareAcronym{API}{
  short=API,
  long=Application Programming Interface
}


\usepackage[
style=numeric,
sorting=none
]{biblatex}
\addbibresource{literature.bib} %Import the bibliography file


\def\contentsname{Contents}
\def\listfigurename{List of Figures}
\def\listtablename{List of Tables}
\def\refname{References}
\def\indexname{Index}
\def\figurename{Obr.}
\def\tablename{Tab.}
\def\partname{Part}
\def\appendixname{Appendix}
\def\abstractname{Abstract}
% IEEE specific names
\def\IEEEkeywordsname{Klíčová slova}
\def\IEEEproofname{Proof}




\title{Název vašeho projektu - článku}


\author{Bc. Jiří Bönsch
        \linebreak
        Faculty of Informatics and Management
        \linebreak
        University of Hradec Kralove,
        \linebreak
        Hradec Kralove, Czech Republic
        \linebreak
        bonscji1@uhk.cz

}

\begin{document}

% make the title area
\maketitle

% As a general rule, do not put math, special symbols or citations
% in the abstract or keywords.
\begin{abstract} % TODO
        abstraktem se rozumí 10 až 15 řádků popisujících stručně obsah vašeho článku. Nejprve popište obecnou problematiku vašeho projektu, následně popište vámi řešený problém a pak čeho jste dosáhli a výsledky spolu s oblastí nasazení/použití.
        hello4.\\
        hello5\\
        hello6.\\
        hello7\\
        hello8.\\
        hello9\\
        hello10\\
        hello11\\
        hello12\\
        hello13\\
        hello14\\
        hello15\\

\end{abstract}

% Note that keywords are not normally used for peerreview papers.
\begin{IEEEkeywords} % TODO
Memory, Computer vision, acquisition of knowledge, educational psychology
\end{IEEEkeywords}


\IEEEpeerreviewmaketitle



\section{INTRODUCTION/ÚVOD}

\subsection{}
Současná doba je spojená s pokrokem technologií.
Technologie prolínají každou část lidského života, není tomu jinak ani v rámci studia.
Elektronické pomůcky nahrazují stále více pomůcek klasických.
Tužka je nahrazena elektronickým perem. Papír nahrazen tabletem.
Vede toto nahrazení však k lepším studijním výsledkům?

\subsection{}
Vědecké studie tuto skutečnost dlouhodobě zkoumají.
Můžeme pozorovat velice jasné návaznosti, kdy pozdější studie upřesnůje výsledky starší studie, nebo starší studii doplňuje o novější aplikace technologie.
Příkladem může být zaměřená na psaní poznámek v akademické prostředí\cite{mightier_pen}. Studie testuje, zda existuje významný rozdíl mezi zapisováním na notebooku a nebo na papír pomocí pera.
Zajímavým výsledkem je zjištění, že psaní pomocí pera je pro zapamatování lepší nežli zapisování poznámek na notebooku.
Hlavním důvodem je skutečnost, že psaní perem je pomalejší, a tak je zapisující nutný volit informace, které zapíše.
Notebook naopak dovoluje doslovné zapisování výkladu.
Zajímavé je, že ani za případu, kdy byl uživatel notebuku seznámen s nevýhodou jeho zapisovací strategie, nevedlo to k významné změně výsledků.

\subsection{}
Velkou mezerou předchozí studie je existence digitálních per, jejichž dopat nebyl zkoumán.
Existuje však novější studie, která se zabývá touto problematikou.
Specificky studie zkoumá, zda využívání klasických per, digitálních a klávesnice vede k lepším výsledků při studiu jednotlivých slov
\cite{advantage_of_handwriting}.
Studie ukazuje, že i pro studování jednotlivých slov je stále nejlepší psaní perem, zajímavé jsou však rozdíly mezy perem digitálním a klasickým.
U subjektů, kteří mají skušenosti s digitálním perem, nebyl vidět značný rozdíl mezi typy pera.
Je tedy možné, že rozdíly mezi typy pera u nezkušených uživatelů mohou být způsobené zvýšenou kognitivní zátěží způsobenou používáním nové pomůcky. To však nic nemění na faktu, že používání kteréhokoli pera bylo pro memorizaci slov výrazně lepší než pouze klávesnice.

\subsection{}
Předchozí sekce mohou vyvolávat dojem, že technologie není vhodná pro usnadnění studia dalšího jazyku. Základ tohoto dojmu je však postaven na velice specifické sféře mnohem složitějšího problému. Vědci si jsou této skutečnosti vědomi, můžeme se proto odkazovat na roky jejich bádání.
Meta-analýza zkoumající více jak 30 studií na toto téma představuje rozsáhlejší pohled na problematiku studia více jazyků\cite{technology_vocab}. Analýza jasně ukazuje, že podpora studia využitím moderní technologie vedla k lepším výsledkům zkušebních testů. Nezáleží ani, zda byl test okamžitě po učení, nebo až s časovým odstupem, vždy je vidět jasná korelace lepších výsledků s používáním moderních technologii. Specificky zajímavé poznatky však vynikly při komparaci výsledků věkových skupin a také typu využívané technologie. Nejlepší výsledky přineslo využití moderních technologií u vysokoškolských studentů.
Nelze však s jistotou říci, zda je tato skutečnost způsobena jejich schopností pracovat s moderními technologiemi nebo jejich znalostní bází anglického jazyka.
Naopak u předškoláků a na prvním stupni základní školy nepřineslo využití moderních technologii značný přínos.
Preference studia na mobilních telefonech je další zajímavý fakt, který je z této analýzy zřejmý.
Za pomoci mobilního telefonu můžeme studovat kdekoli a to v porovnání se studiem na stolních počítačích ve třídě k vedlo k lepším výsledkům.

\subsection{}
Je zřejmé, že použití moderních technologií má smysl.
Nelze však bezmyšlenkovitě vynucovat používání.
Můžeme uvažovat o nutnost zaměřit nasazování moderních technologií pouze do specifických odvětví studia jazyků, kde povedou ke zvýšení efektivity.



\subsection{OBSAH}
V úvodu definovat oblast bádání (problematiky) a postupně se dostat k tomu, kde je problém (v závěrečné části úvodu).
\subsection{ROZSAH}
Tato kapitola by měla mít rozsah minimálně do konce první strany dokumentu. Maximálně však do poloviny levého sloupce na druhé straně.
\subsection{CITACE}
V této kapitole budou alespoň 3, lépe však 5 odkazů na literaturu, vztahující se k popisované problematice, aby bylo z textu patrné, že se jedná o aktuální téma.
Minimálně je třeba nalézt 3 články z impaktovaných časopisů ISI WOK JCR, nebo z konferencí indexovaných ISI WOK CPCI. Najdete na stránce www.isiknowledge.com ale pouze pokud jste v rámci UHK (tedy buď na PC v budovách UHK, nebo přes VPN).



\section{PROBLEM DEFINITION/ DEFINICE PROBLÉMU}
\subsection{OBSAH}
V této kapitole je třeba definovat problém a ukázat alespoň tři řešení (lépe 5) od někoho jiného (formou odstavce shrnujícího přístup dotyčného (3 až 5 řádků)). Kapitola by měla končit konstatováním, že žádný z přístupů neřeší definovaný problém tak, jak by bylo třeba (jak bychom potřebovali my) a proto je třeba najít nový způsob (ten náš), o kterém se bude pojednávat v další kapitole.
\subsection{ROZSAH}
Tato kapitola by měla mít rozsah cca 1 stranu.
\subsection{CITACE}
V této kapitole budou alespoň 3, lépe však 5 odkazů na literaturu, vztahující se k popisované problematice, aby bylo z textu patrné, že se jedná o aktuální téma.




\section{NEW SOLUTION / NOVÉ ŘEŠENÍ}

\subsection{OBSAH}
V této kapitole je třeba přesně popsat nový způsob řešení a to včetně nutné teorie, která s tím souvisí. 
\subsection{ROZSAH}
Rozsahem je minimálně 1 strana a max. 2 strany.




\section{IMPLEMENTATION / IMPLEMENTACE ŘEŠENÍ}
\subsection{OBSAH}
Tato kapitola by měla pojednávat o praktické implementaci nového řešení. Tedy jak dojít od teorie k implementaci a jak jsme to řešili my (vy).
\subsection{ROZSAH}
Rozsah je min. 1 strana, maximálně 2 strany.





\section{TESTING OF DEVELOPED APPLICATION / TESTOVÁNÍ VYVINUTÉ APLIKACE - ŘEŠENÍ}
\subsection{OBSAH}
Zde musí být definice, jak bude testováno a co má být přesně výsledkem.
Vlastní testování a výsledky formou tabulek budou v podkapitole
Zhodnocení výsledků testování je nejlépe slovně (zhodnocení předchozích tabulek) a pak jedna tabulka s přehledem řešení od jiných autorů s tím novým řešením (mělo by se ukázat, že to nové řešení je nejlepší)
\subsection{ROZSAH}
Rozsah je 1strana. \ac{SSIM}



\section{CONCLUSIONS / ZÁVĚRY}
Tady už se vyjádřit jen k tomu, že se podařilo najít (definovat) nový přístup k řešení problému a že byl i prakticky ověřen na modelovém případě. 
Dobré je také diskutovat využitelnost nového řešení jak v aktuální oblasti problému (nejlépe včetně finančních či časových úspor), tak i v dalších oblastech (alespoň nastínit).
Rozsah závěru je minimálně 10 řádků, maximálně 20 řádků.\ac{MSE}\cite{einstein}

\ac{SSIM}


% seznam zdrojů
\printbibliography

%  seznam zkratek
\printacronyms


% that's all folks
\end{document}
