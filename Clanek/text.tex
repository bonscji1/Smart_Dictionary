\documentclass[journal]{IEEEtran}


\usepackage[czech]{babel}


% DOPLNIJICI BALICKY
\usepackage[utf8]		%	Kódování zdrojových souborů je Windows-1250
	{inputenc}					% Balíček pro nastavení kódování zdrojových souborů
\usepackage{cmap} 		% Balíček cmap zajišťuje, že PDF vytvořené `pdflatexem' je
											% plně "prohledávatelné" a "kopírovatelné"

\usepackage{amsmath}
\usepackage{caption}
\usepackage{dsfont}
\usepackage{layouts}
\usepackage{graphicx}
\usepackage{hyperref}
\usepackage{subcaption}
\usepackage{acro}

% zkratky
\DeclareAcronym{MSE}{
  short=MSE,
  long=mean squared error
}

\DeclareAcronym{PSNR}{
  short=PSNR,
  long=peak signal-to-noise ratio
}

\DeclareAcronym{SSIM}{
  short=SSIM,
  long=structural similarity
}

\DeclareAcronym{OCR}{
  short=OCR,
  long=Optical Character Recognition
}

\DeclareAcronym{IDE}{
  short=IDE,
  long=Integrated Development Environment
}

\DeclareAcronym{API}{
  short=API,
  long=Application Programming Interface
}


\usepackage[
style=numeric,
sorting=none
]{biblatex}
\addbibresource{literature.bib} %Import the bibliography file


\def\contentsname{Contents}
\def\listfigurename{List of Figures}
\def\listtablename{List of Tables}
\def\refname{References}
\def\indexname{Index}
\def\figurename{Obr.}
\def\tablename{Tab.}
\def\partname{Part}
\def\appendixname{Appendix}
\def\abstractname{Abstract}
% IEEE specific names
\def\IEEEkeywordsname{Klíčová slova}
\def\IEEEproofname{Proof}




\title{Název vašeho projektu - článku}


\author{Bc. Jiří Bönsch
        \linebreak
        Faculty of Informatics and Management
        \linebreak
        University of Hradec Kralove,
        \linebreak
        Hradec Kralove, Czech Republic
        \linebreak
        bonscji1@uhk.cz

}

\begin{document}

% make the title area
\maketitle

% As a general rule, do not put math, special symbols or citations
% in the abstract or keywords.
\begin{abstract} % TODO
        abstraktem se rozumí 10 až 15 řádků popisujících stručně obsah vašeho článku. Nejprve popište obecnou problematiku vašeho projektu, následně popište vámi řešený problém a pak čeho jste dosáhli a výsledky spolu s oblastí nasazení/použití.
        hello4.\\
        hello5\\
        hello6.\\
        hello7\\
        hello8.\\
        hello9\\
        hello10\\
        hello11\\
        hello12\\
        hello13\\
        hello14\\
        hello15\\

\end{abstract}

% Note that keywords are not normally used for peerreview papers.
\begin{IEEEkeywords} % TODO
Memory, Computer vision, acquisition of knowledge, educational psychology
\end{IEEEkeywords}


\IEEEpeerreviewmaketitle



\section{INTRODUCTION/ÚVOD}

\subsection{}
Současná doba je spojená s pokrokem technologií.
Technologie prolínají každou část lidského života, není tomu jinak ani v rámci studia.
Elektronické pomůcky nahrazují stále více pomůcek klasických.
Tužka je nahrazena elektronickým perem. Papír nahrazen tabletem.
Podobná nahrazení jsou ve všech aspektech studia, je však nutné si položit otázku.
Vede toto nahrazení  k lepším studijním výsledkům?

\subsection{}
Vědecké studie tuto skutečnost dlouhodobě zkoumají.
Můžeme pozorovat velice jasné návaznosti, kdy pozdější studie upřesnůje výsledky starší studie, nebo starší studii doplňuje o novější aplikace technologie.
Příkladem může být zaměřená na psaní poznámek v akademické prostředí\cite{mightier_pen}. Studie testuje, zda existuje významný rozdíl mezi zapisováním na notebooku a nebo na papír pomocí pera.
Zajímavým výsledkem je zjištění, že psaní pomocí pera je pro zapamatování lepší nežli zapisování poznámek na notebooku.
Hlavním důvodem je skutečnost, že psaní perem je pomalejší, a tak je zapisující nutný volit informace, které zapíše.
Notebook naopak dovoluje doslovné zapisování výkladu.
Zajímavé je, že ani za případu, kdy byl uživatel notebuku seznámen s nevýhodou jeho zapisovací strategie, nevedlo to k významné změně výsledků.

\subsection{}
Velkou mezerou předchozí studie je existence digitálních per, jejichž dopat nebyl zkoumán.
Existuje však novější studie, která se zabývá touto problematikou.
Specificky studie zkoumá, zda využívání klasických per, digitálních a klávesnice vede k lepším výsledků při studiu jednotlivých slov
\cite{advantage_of_handwriting}.
Studie ukazuje, že i pro studování jednotlivých slov je stále nejlepší psaní perem, zajímavé jsou však rozdíly mezy perem digitálním a klasickým.
U subjektů, kteří mají skušenosti s digitálním perem, nebyl vidět značný rozdíl mezi typy pera.
Je tedy možné, že rozdíly mezi typy pera u nezkušených uživatelů mohou být způsobené zvýšenou kognitivní zátěží způsobenou používáním nové pomůcky. To však nic nemění na faktu, že používání kteréhokoli pera bylo pro memorizaci slov výrazně lepší než pouze klávesnice.

\subsection{}
Předchozí sekce mohou vyvolávat dojem, že technologie není vhodná pro usnadnění studia dalšího jazyku. Základ tohoto dojmu je však postaven na velice specifické sféře mnohem složitějšího problému. Vědci si jsou této skutečnosti vědomi, můžeme se proto odkazovat na roky jejich bádání.
Meta-analýza zkoumající více jak 30 studií na toto téma představuje rozsáhlejší pohled na problematiku studia více jazyků\cite{technology_vocab}. Analýza jasně ukazuje, že podpora studia využitím moderní technologie vedla k lepším výsledkům zkušebních testů. Nezáleží ani, zda byl test okamžitě po učení, nebo až s časovým odstupem, vždy je vidět jasná korelace lepších výsledků s používáním moderních technologii. Specificky zajímavé poznatky však vynikly při komparaci výsledků věkových skupin a také typu využívané technologie. Nejlepší výsledky přineslo využití moderních technologií u vysokoškolských studentů.
Nelze však s jistotou říci, zda je tato skutečnost způsobena jejich schopností pracovat s moderními technologiemi nebo jejich znalostní bází anglického jazyka.
Naopak u předškoláků a na prvním stupni základní školy nepřineslo využití moderních technologii značný přínos.
Preference studia na mobilních telefonech je další zajímavý fakt, který je z této analýzy zřejmý.
Za pomoci mobilního telefonu můžeme studovat kdekoli a to v porovnání se studiem na stolních počítačích ve třídě k vedlo k lepším výsledkům.

\subsection{}
Je zřejmé, že použití moderních technologií má smysl.
Nelze však bezmyšlenkovitě vynucovat používání.
Můžeme uvažovat o nutnost zaměřit nasazování moderních technologií pouze do specifických odvětví studia jazyků, kde povedou ke zvýšení efektivity.

\section{PROBLEM DEFINITION/ DEFINICE PROBLÉMU}

\subsection{Problematika}
Jak již bylo zmíněno, pro maximalizaci efektivity studia by bylo ideální využívat jak výhody zapisování materiálu perem, tak výhody digitální podoby materiálu. Pokud se podíváme blíže na ulohu studování slovní zásoby jazyka, můžeme jasně identifikovat tyto dva protilehlé faktory. Pro lepší zapamatování slov je lepší zapisovat je perem, naopak pro možnosti opakování a také dodatečných pomůcek v podobě optimalizace a preferování zaměření na složitá slova je ideální digitální forma. Cíl je jasný, najít nebo vytvořit řešení, která umožní využívat tyto dvě nekompatibilní metodologie současně.

\subsection{Průzkum existujících řešení}
Tato práce není první, která se danou problematikou zabývá.
Hlubší průzkum problematiky vede k již existujícím řešením v podobě digitalizace psaného textu.
Je možné nalézt hned několik možností, jak digitalizace dosáhnout.
Existence několika srovnatelných možností však poukazuje na fakt, že neexistuje jedno perfektní řešení, které by bylo jasně lepší nežli všechna ostatní a tak uživateli preferováno.
Ne každý postup je vhodný pro každáho uživatele, ať už se jedná o problematiku ceny, požadavků na fyzická zařízení nebo dokonce přesnost převodu či osobní principy.\cite{aarp_digitalization, popupalr_science_digitalization_with_pens}

\paragraph{Smart pera}
Prvním možným řešením jsou systémy využívající smart pera.
Tento systém má 3 části, již zmíněné smart pero, speciální tečkovaný blok a apklikaci.
Smart pero je schopno klasicky zapisovat a při tomto zápisu snímá pohyby ruky, tečkovaný blok pak využívá k přesnému zaznamenávání umístění psaného textu na stránku. Všechna tato data jsou pak zaslána aplikaci, která vytváří přesnou digitální kopii psaného textu. Aplikace je také schopna ručně psanou digitální kopii převést do klasického formátu microsoft word. Hlavní výhodou tohoto řešení je značná přesnost, ne každý je však ochotný platit prémium za takto specifický use-case.\cite{popupalr_science_digitalization_with_pens}

\paragraph{Smartphone}
Dalším řešením je využívání smartphonu, který v dnešní době vlastní většina populace.
Apple od verze iOS 15 zabudoval do operačního systému aplikaci Live Text, která uživateli dovoluje zvolit a poté zvolený text vykopírovat.
Android  tuto funkcionalitu v operačním systému zabudovanou nemá. tuto skutečnost lze napravit pomocí aplikace Google Keep však dovoluje z obrázku také extrahovat raw text.\cite{aarp_digitalization, google_keep}
Výhodou tohoto řešení je dostupnost, nevýhodou může být přesnost ale také složitější postup následného využívání vykopírovaného textu.

\paragraph{Další aplikace využívající OCR}
Pro kompletnost je nutné zmínit další aplikace, které využívající \ac{OCR} pro digitalizaci psaného textu z obrázků. Příkladem Evernote\cite{evernote}, Microsoft OneNote\cite{onenote} nebo Pen to Print\cite{pen_to_print}.
I zde je bohužel problém přesnosti a následného využití získaného digitalizovaného textu.


\subsection{Závěrečné konstatování}
Je nutné konstatovat, že žádný z dříve zmíněných přístupů neřeší  kompletní problém, který inspiroval tuto práci. Řešená je pouze  prvotní část problému, převod psaného slova, ale digitalizovaná podoba není dále nijak využívána a není tedy platné brát tyto možnosti jako řešení problematiky učení slovní zásoby.
Jako jednoduché řešení by se mohla zdát nástavba některé předchozí texhnologie, využít digitalizaci a výstup dále zpracovávat. Narážíme však na standartní problémy při pokusech nastavovat již zavedenou a uzavřenou aplikaci. Nejen že není  kontrola nad fungováním digitalizační aplikace, ale také jakékoli změny ve výstupu či funčnosti mohou zamezit fungování takto vytvořené nástavby.
Je tedy nutné vytvořit nový způsob, kde budou tyto faktory pod kontrolou a pouze využít již existující řešení pouze inspiraci.

\section{NEW SOLUTION / NOVÉ ŘEŠENÍ}

\subsection{Teoretické zpracování problematiky}
Jak již bylo zmíněno v předchozí sekci, průzkum tématiky zcela jasně vede k využití technologie \ac{OCR} pro digitalizaci textu.
\ac{OCR} je process který převádí obrázek, na kterém se nachází text, do textu zpracovatelného strojem.
Hlavním využitím této technologie je digitalizace dokumentů pro další zpracování. V dnešní době je snadné vytvořit scan legálně platné smlouvy smlouvy nebo vyúčtování, což je vhodné pro dlouhodobé skladování at již z pohledu zachovatelnosti nebo zabraného místa.
Takto naskenované dokumenty se v budoucnu velmi těžko zpracovávají, proto je výhodné provézt převod do klasické textové podoby pokud s takovýmto dokumentem chceme v budoucnu nadále pracovat. Další výhodou je taky možnost vyhledávání v textu, která je stále rychlejší a přesnější než vyhledávání na obrázcích.\cite{amazon_ocr}

\subsubsection{metodika OCR}
Ve většině případů můžeme ve zpracování dat pomocí \ac{OCR} identifikovat několik základních kroků.\cite{amazon_ocr}

\begin{itemize}
 \item Zisk obrázku
 \item Předzpracování obrázku
 \item Rozpoznání textu
 \item Následné zpracování
\end{itemize}

\paragraph{Zisk obrázku}
Pro \ac{OCR} potřebujeme obrázek s rozeznatelným textem, který nejčastěji získáme vyfocením nebo lépe naskenováním požadovaných dat.

\paragraph{Předzpracování}
Obrázek si systém předzpracovává z důvodu vyčištěí a odstranění nejčastějších chyb obrazu.
Tento krok je dlůežitý jelikož vede ke zvýšení přesnosti \ac{OCR}.
Mezi nejčastější úkony předzpracování jsou změny náklonu, začištování hran nebo vyčištění nežádoucích obrazců, boxů a linií.

\paragraph{Rozpoznání textu}
Samotný algoritmus \ac{OCR} může být 2 typů, \textit {Pattern matching} nebo \textit{Feature extraction}.
\textit{Pattern matching} funguje izolováním jednotlivých charakterů textu a nasledným porovnáváním tohoto charakteru s vnitřní databází charakterů. Tento systém funguje pouze tehdy, pokud s požadovanou přesností dokáže k izolovanému charakteru najít ve vnitřní databázi skupinu, do které zapadá a tak určit, o který charakter se jedná.

\textit{Feature extraction} pokračuje v rozkládání izolovaných charakterů na jednotlivé linie a tahy daného charakteru. Dalším krokem tohoto postupu je opět komparace s interní databází pro nalezení nejbližší schody s interním charakterem, který je výslekdem klasifikace.

\paragraph{Následné zpracování}
Po provedení předchozích kroků provede systém konverzi extraktovaného textu do digitalizovaného souboru, a to v závislosti na nastavených požadavcích.

\subsubsection{databáze PostgreSQL}
V rámci ukládání je nutné využít databázi, v případě této práce byla zvolena PostgreSQL.
PostgreSQL je \textit{open-source} objektově relační databáze, která je aktivně vyvíjená více jak 35 let.
Tato databáze je proslulá svojí spolehnovistí, rozsáhlou funkcionalitou a výkonem.
Dalším faktorem volby této databáze je jednoduchost nasazení za použítí \textit{Docker containeru} a rozsáhlá dokumentace pro případné řešení problémů.\cite{postgre}

\subsection{Představa výsledného řešení}
V předchozím textu již byly identifikovány hlavní kroky nutné pro vyřešení stanoveného problému této práce.
Prvním krokem je převést ručně psaný text ve slovníkové podobě, tedy \textit{španělské slovo} - \textit{český význam}, do digitální podoby za pomoci \ac{OCR}.
Dalším krokem je uložení takto získané digitální podoby slovníkového záznamu do databázového systému, specificky PostgreSQL.\cite{postgre}
Pokud tento záznam již existuje, je nutné změnit jeho prioritu naučení.
Tento záznam jsme se v minulosti nenaučili a je tedy nutné mu do budoucna věnovat větší pozornost při studiu.
Pro následné studium, například pomocí pomodoro techniky, je také dobré zachvávat datum, kdy byl záznam vytvořen.


\section{IMPLEMENTATION / IMPLEMENTACE ŘEŠENÍ}

\subsection{Popis Implementace}
Pro implementaci \ac{OCR} byly vybrány 2 hlavní technologie, Tesseract\cite{tesseract} a EasyOCR\cite{easy_ocr}.
Dále byla pro komparaci vybrána technologie digitalizace Pen to Print\cite{pen_to_print}.
Tato technologie není na rozdíl od technologií Tesseract a EasyOCR využitelná pro praktické provedení řešení, je to však placená technologie specificky vyvinutá a prodávaná pro převod ručně psaného textu do digitální formy. Je tedy vhodná jako benchmark pro komparaci vytvořeného řešení.

\subsubsection{Tesseract}
Tessetact je \ac{OCR} řešení původně vyvíjené v Hewlett-packard laboratořích v roce 1985, později vydané jako \textit{open-source} v roce 2005 a dále jeho vývoj převzal Google od roku 2006.
Je zřejmé, že toto řešení má bohatou historii a je možná nejznámější \ac{OCR} engine.
V současné době je tento projekt ve verzi 5, v základní verzi podporuje více jak 100 jazyků, je schopný zpracovávat několik typů vstupních obrázků a díky své komunitě na něj existuje mnoho nástavem pro specifická řešení.
V této práci bude Tesseract využíván pomocí python wrappru zvaného \textit{PyTesseract}.

\subsubsection{EasyOCR}
EasyOCR je \textit{open-source} python modul pro extrakci textu z obrázků, schopné zpracovat přirozený i zhuštěný text v dokumentech.
EasyOCR v době psaní  podporoje přes 80 jazyků s cílem další expanze.
Dalším milníkem tohoto projektu je, dle oficiální github stránky, zpracování ručně psaného textu, v následujícím textu je popsána současná situace týkající se tohoto milníku.\cite{easy_ocr}

\subsubsection{Pen to Print}

kontrola

\subsection{Omezení/nevytvořitelnost}
co jsem dělal, proč to nejde
proč to nejde
kde to selhává

\subsection{Opuštění dalších částí řešení}
není nutné dělat další části řešení, pokud nefunguje ani první krok,
více prozkoumat limitace prvního kroku


\subsection{OBSAH}
Tato kapitola by měla pojednávat o praktické implementaci nového řešení. Tedy jak dojít od teorie k implementaci a jak jsme to řešili my (vy).
\subsection{ROZSAH}
Rozsah je min. 1 strana, maximálně 2 strany.





\section{TESTING OF DEVELOPED APPLICATION / TESTOVÁNÍ VYVINUTÉ APLIKACE - ŘEŠENÍ}

\subsection{testování možností}

\subsection{výstupy a jejich porovnání}

\subsection{OBSAH}
Zde musí být definice, jak bude testováno a co má být přesně výsledkem.
Vlastní testování a výsledky formou tabulek budou v podkapitole
Zhodnocení výsledků testování je nejlépe slovně (zhodnocení předchozích tabulek) a pak jedna tabulka s přehledem řešení od jiných autorů s tím novým řešením (mělo by se ukázat, že to nové řešení je nejlepší)
\subsection{ROZSAH}
Rozsah je 1strana. \ac{SSIM}



\section{CONCLUSIONS / ZÁVĚRY}
\subsection{závěr}
nejde to
proč to nejde
shrnutí omezení metodiky
návrh sledovat problematiku
návrh vytvořit učící model na psané písmo pokud se tomu dají větší prostředky

Tady už se vyjádřit jen k tomu, že se podařilo najít (definovat) nový přístup k řešení problému a že byl i prakticky ověřen na modelovém případě. 
Dobré je také diskutovat využitelnost nového řešení jak v aktuální oblasti problému (nejlépe včetně finančních či časových úspor), tak i v dalších oblastech (alespoň nastínit).
\subsection{ROZSAH}
Rozsah závěru je minimálně 10 řádků, maximálně 20 řádků.\ac{MSE}\cite{einstein}



% seznam zdrojů
\printbibliography

%  seznam zkratek
\printacronyms


% that's all folks
\end{document}
